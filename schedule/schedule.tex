\batchmode
\makeatletter
\def\input@path{{/accounts/gen/vis/paciorek/staff/workshops/r-bootcamp-2013/schedule//}}
\makeatother
\documentclass[12pt]{article}\usepackage{graphicx, color}
%% maxwidth is the original width if it is less than linewidth
%% otherwise use linewidth (to make sure the graphics do not exceed the margin)
\makeatletter
\def\maxwidth{ %
  \ifdim\Gin@nat@width>\linewidth
    \linewidth
  \else
    \Gin@nat@width
  \fi
}
\makeatother

\definecolor{fgcolor}{rgb}{0.2, 0.2, 0.2}
\newcommand{\hlnumber}[1]{\textcolor[rgb]{0,0,0}{#1}}%
\newcommand{\hlfunctioncall}[1]{\textcolor[rgb]{0.501960784313725,0,0.329411764705882}{\textbf{#1}}}%
\newcommand{\hlstring}[1]{\textcolor[rgb]{0.6,0.6,1}{#1}}%
\newcommand{\hlkeyword}[1]{\textcolor[rgb]{0,0,0}{\textbf{#1}}}%
\newcommand{\hlargument}[1]{\textcolor[rgb]{0.690196078431373,0.250980392156863,0.0196078431372549}{#1}}%
\newcommand{\hlcomment}[1]{\textcolor[rgb]{0.180392156862745,0.6,0.341176470588235}{#1}}%
\newcommand{\hlroxygencomment}[1]{\textcolor[rgb]{0.43921568627451,0.47843137254902,0.701960784313725}{#1}}%
\newcommand{\hlformalargs}[1]{\textcolor[rgb]{0.690196078431373,0.250980392156863,0.0196078431372549}{#1}}%
\newcommand{\hleqformalargs}[1]{\textcolor[rgb]{0.690196078431373,0.250980392156863,0.0196078431372549}{#1}}%
\newcommand{\hlassignement}[1]{\textcolor[rgb]{0,0,0}{\textbf{#1}}}%
\newcommand{\hlpackage}[1]{\textcolor[rgb]{0.588235294117647,0.709803921568627,0.145098039215686}{#1}}%
\newcommand{\hlslot}[1]{\textit{#1}}%
\newcommand{\hlsymbol}[1]{\textcolor[rgb]{0,0,0}{#1}}%
\newcommand{\hlprompt}[1]{\textcolor[rgb]{0.2,0.2,0.2}{#1}}%

\usepackage{framed}
\makeatletter
\newenvironment{kframe}{%
 \def\at@end@of@kframe{}%
 \ifinner\ifhmode%
  \def\at@end@of@kframe{\end{minipage}}%
  \begin{minipage}{\columnwidth}%
 \fi\fi%
 \def\FrameCommand##1{\hskip\@totalleftmargin \hskip-\fboxsep
 \colorbox{shadecolor}{##1}\hskip-\fboxsep
     % There is no \\@totalrightmargin, so:
     \hskip-\linewidth \hskip-\@totalleftmargin \hskip\columnwidth}%
 \MakeFramed {\advance\hsize-\width
   \@totalleftmargin\z@ \linewidth\hsize
   \@setminipage}}%
 {\par\unskip\endMakeFramed%
 \at@end@of@kframe}
\makeatother

\definecolor{shadecolor}{rgb}{.97, .97, .97}
\definecolor{messagecolor}{rgb}{0, 0, 0}
\definecolor{warningcolor}{rgb}{1, 0, 1}
\definecolor{errorcolor}{rgb}{1, 0, 0}
\newenvironment{knitrout}{}{} % an empty environment to be redefined in TeX

\usepackage{alltt}
\usepackage[T1]{fontenc}
\usepackage[latin9]{inputenc}
\usepackage[letterpaper]{geometry}
\geometry{verbose,tmargin=1in,bmargin=1in,lmargin=1in,rmargin=1in}
\usepackage{setspace}
\onehalfspacing

\makeatletter
%%%%%%%%%%%%%%%%%%%%%%%%%%%%%% User specified LaTeX commands.
\usepackage{/accounts/gen/vis/paciorek/latex/paciorek-asa,times,graphics}
\input{/accounts/gen/vis/paciorek/latex/paciorekMacros}
%\renewcommand{\baselinestretch}{1.5}
\usepackage[unicode=true]{hyperref}
\hypersetup{unicode=true, pdfusetitle,
bookmarks=true,bookmarksnumbered=false,bookmarksopen=false,
 breaklinks=false,pdfborder={0 0 1},backref=false,colorlinks=true,}

\makeatother
\IfFileExists{upquote.sty}{\usepackage{upquote}}{}

\begin{document}


\title{R bootcamp - August 2013: Syllabus/schedule}

\maketitle

\subsection*{}

Unless otherwise noted, sessions are 60-70 minutes long, including
time for work on breakout problems.
\begin{itemize}
\item Day 1 morning (8:30-12:30) (learning R)

\begin{itemize}
\item Session 0: introduction, what is R, starting R, why R? why not R?
(Chris P.) (10 minutes)
\item Session 1: basics of R, with Rstudio (Chris P.)

\begin{itemize}
\item R as a calculator
\item helpful shortcuts: tab-complete, up arrow, Ctrl-\{up arrow\}
\item vectors and indexing and subset assignment
\item some basic functions; help()
\item vectorized calculations, comparisons
\item basic R objects: vectors, matrices, dataframes, lists
\item managing R objects, the R workspace
\item basic graphics
\item breakout problems  
\end{itemize}
\item Session 2: Working with data (Chris P.)

\begin{itemize}
\item dataframes/matrices
\item attributes, missing values and factors
\item subsetting
\item strings
\item reading/writing data; working directory, foreign package
\item breakout problems 
\end{itemize}
\item Break (20 minutes)
\item Session 3: Calculations (Chris P.)

\begin{itemize}
\item vectorized calculations and efficiency
\item apply, lapply
\item tabulation, stratified analyses, aggregation, merging data
\item breakout problems 
\end{itemize}
\end{itemize}
\item Lunch (on your own) (12:30-1:30)
\item Day 1 afternoon (1:30-5:00) (programming and real-world work)

\begin{itemize}
\item Session 4: R resources (Chris P.) (30 minutes)

\begin{itemize}
\item packages - installing, loading, namespaces
\item getting R help online 
\end{itemize}
\item Session 5: programming in R (Jacob)

\begin{itemize}
\item loops, if-else
\item writing your own functions, function arguments, functions as objects
\item basic scoping and environments
\item breakout problems
\end{itemize}
\item Break (20 minutes)
\item Session 6: doing useful stuff (Chris K.)

\begin{itemize}
\item stratified analyses: groupwise operations (see plyr: subset, mutate,
summarise, arrange); split-apply-combine
\item reshape
\item regression, GLMs 
\item breakout problems/homework 
\end{itemize}
\end{itemize}
\item Day 2 morning (9-12:30) (more real-world work)

\begin{itemize}
\item Session 7: Some core tools (Chris P.) (45 minutes)

\begin{itemize}
\item go over homework
\item smoothing
\item optimization
\item simulation, sample()
\item dates and times
\item breakout 
\end{itemize}
\item Session 8: Graphics (Chris K.)

\begin{itemize}
\item exporting graphics (vector/raster formats)
\item lattice graphics
\item ggplot2
\item breakout problems 
\end{itemize}
\item Break (20 minutes)
\item Session 9: Workflows, coding practices, and project management (Jarrod)

\begin{itemize}
\item scripting, source(); separating data, code, figures
\item R in batch mode and command line mode
\item timing, memory use, debugging
\item reproducible research with knitr, Rmd
\item version control for code and data; Git
\item breakout problems
\end{itemize}
\end{itemize}
\item Lunch (on your own) (12:30-1:30)
\item Day 2 afternoon (1:30-4:30) (more advanced topics) 

\begin{itemize}
\item Session 10: quick tastes of advanced topics (Chris P.)

\begin{itemize}
\item OOP (S3, S4, ReferenceClasses)
\item computing on the language (using R to write and evaluate R code)
\item errors and try-catch
\item encodings
\item working with databases
\item breakout problems
\end{itemize}
\item Break + Feedback forms (20 minutes)
\item Session 11: parallel processing (Chris P.)

\begin{itemize}
\item foreach
\item parApply and variants
\item RNG issues
\item breakout problems 
\end{itemize}
\item Session 12: Wrapping up (Chris P.) (15 minutes)

\begin{itemize}
\item R inconsistencies and different ways to do things 
\item Where to learn more (campus and non-campus resources)\end{itemize}
\end{itemize}
\end{itemize}


\end{document}
